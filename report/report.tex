\documentclass[acmtog]{acmart}
\usepackage{graphicx}
\usepackage{subfigure}
\usepackage{natbib}
\usepackage{listings}
\usepackage{bm}
\usepackage{amsmath}

\definecolor{blve}{rgb}{0.3372549 , 0.61176471, 0.83921569}
\definecolor{gr33n}{rgb}{0.29019608, 0.7372549, 0.64705882}
\makeatletter
\lst@InstallKeywords k{class}{classstyle}\slshape{classstyle}{}ld
\makeatother
\lstset{language=C++,
	basicstyle=\ttfamily,
	keywordstyle=\color{blve}\ttfamily,
	stringstyle=\color{red}\ttfamily,
	commentstyle=\color{magenta}\ttfamily,
	morecomment=[l][\color{magenta}]{\#},
	classstyle = \bfseries\color{gr33n}, 
	tabsize=2
}
\lstset{basicstyle=\ttfamily}

% Title portion
\title{Assignment 3:\\ {Basic Ray Tracing}} 

\author{Name: Xu WeiHan\quad \\ student number: 2023533008\
\\email: \quad \texttt{xuwh2023@shanghaitech.edu.cn}}

% Document starts
\begin{document}
\maketitle

\vspace*{2 ex}

\section{Introduction}

In this assignment. I implemented the following tasks:

\begin{itemize}
    \item  Compile the source code and configure the language server environment.  (MUST)
    \item  Implement ray-triangle intersection functionality. (MUST)
    \item  Implement ray-AABB intersection functionality.(MUST)
    \item Implement the BVH (Bounding Volume Hierarchy) construction. (MUST)
    \item Implement the IntersectionTestIntegrator and PerfectRefraction material for basic ray tracing validation, handing refractive and solid surface interactions (MUST)
    \item  Implement a direct lighting function with diffuse BRDF and shadow testing. (MUST)
    \item  Implement anti-aliasing via multi-ray sampling per pixel within a sub-pixel aperture.(MUST)
    \item  Implement support for multiple light sources. (OPTIONAL)
    \item  Implement environment lighting via environment maps. (OPTIONAL)
    \item  Investigate texture mapping support in the code base, utilize them on an exmaple scene and understand the texture mapping (OPTIONAL)
\end{itemize}

\section{Implementation Details}
\subsection{Ray-Triangle Intersection}

The ray-triangle intersection is implemented in the \texttt{TriangleIntersect} function (lines 73-152 in \texttt{src/accel.cpp}) using the \textbf{Möller-Trumbore algorithm}.

The implementation follows these key steps:
\begin{enumerate}
    \item Compute edge vectors: $\mathbf{e}_1 = \mathbf{v}_1 - \mathbf{v}_0$ and $\mathbf{e}_2 = \mathbf{v}_2 - \mathbf{v}_0$
    \item Calculate determinant: $\text{det} = \mathbf{e}_1 \cdot (\mathbf{d} \times \mathbf{e}_2)$, reject if $|\text{det}| < 10^{-8}$
    \item Compute barycentric coordinates $u$ and $v$, verify $u, v \geq 0$ and $u + v \leq 1$
    \item Calculate intersection distance $t$ and check if $t \in [t_{\min}, t_{\max}]$
    \item Update surface interaction with barycentric coordinates $(1-u-v, u, v)$
\end{enumerate}

The implementation uses double precision for intermediate calculations to ensure numerical stability, then converts results back to single precision.

\subsection{Ray-AABB Intersection}

The ray-AABB (Axis-Aligned Bounding Box) intersection is implemented in the \texttt{AABB::intersect} method (lines 32-65 in \texttt{src/accel.cpp}) using the \textbf{slab method}.

The implementation follows these steps:
\begin{enumerate}
    \item Compute intersection times with all three pairs of parallel planes:
    \begin{equation}
        \mathbf{t}_1 = (\mathbf{low} - \mathbf{o}) \cdot \mathbf{d}^{-1}, \quad \mathbf{t}_2 = (\mathbf{upper} - \mathbf{o}) \cdot \mathbf{d}^{-1}
    \end{equation}
    where $\mathbf{d}^{-1}$ is the safe inverse direction of the ray
    
    \item Calculate near and far intersection points for each axis:
    \begin{equation}
        \mathbf{t}_{\text{near}} = \min(\mathbf{t}_1, \mathbf{t}_2), \quad \mathbf{t}_{\text{far}} = \max(\mathbf{t}_1, \mathbf{t}_2)
    \end{equation}
    
    \item Determine entry and exit times:
    \begin{equation}
        t_{\text{enter}} = \max(t_{\text{near}}), \quad t_{\text{exit}} = \min(t_{\text{far}})
    \end{equation}
    
    \item Check validity: if $t_{\text{enter}} > t_{\text{exit}}$, no intersection exists
    
    \item Return $t_{\text{enter}}$ and $t_{\text{exit}}$ via output pointers
\end{enumerate}

This method efficiently tests ray-box intersection by treating the AABB as the intersection of three slabs perpendicular to the coordinate axes.

\subsection{BVH (Bounding Volume Hierarchy) Construction}

The BVH construction is implemented in the \texttt{BVHTree::build} method (lines 122-220 in \texttt{include/rdr/bvh\_tree.h}) using a recursive top-down approach.

The implementation follows these key steps:
\begin{enumerate}
    \item \textbf{Pre-compute Bounding Box}: Calculate the AABB that encompasses all primitives in the current span $[\text{span\_left}, \text{span\_right})$
    
    \item \textbf{Stopping Criteria}: Create a leaf node if either:
    \begin{itemize}
        \item Recursion depth $\geq$ 22 (CUTOFF\_DEPTH), or
        \item Number of primitives $\leq$ 1
    \end{itemize}
    
    \item \textbf{Splitting Strategy}: Use median split heuristic:
    \begin{itemize}
        \item Select splitting dimension as the longest axis of the bounding box
        \item Partition primitives at the median using \texttt{std::nth\_element}
        \item Sort by centroid position along the chosen axis
    \end{itemize}
    
    \item \textbf{Recursive Construction}: Build left and right subtrees:
    \begin{equation}
        \text{split} = \text{span\_left} + \frac{\text{count}}{2}
    \end{equation}
    Left subtree: $[\text{span\_left}, \text{split})$, Right subtree: $[\text{split}, \text{span\_right})$
    
    \item \textbf{Node Organization}: Store internal nodes in a separate array with pointers to children and associated AABBs
\end{enumerate}

The BVH accelerates ray tracing by hierarchical pruning: rays only traverse subtrees whose AABBs intersect with the ray, significantly reducing intersection tests.

\subsection{IntersectionTestIntegrator and PerfectRefraction Material}

This implementation provides basic ray tracing validation with support for refractive and solid surface interactions.

\subsubsection{IntersectionTestIntegrator}

Implemented in \texttt{src/integrator.cpp} (lines 28-213), this integrator traces rays through the scene and handles different material types:

\begin{enumerate}
    \item \textbf{Ray Generation}: For each pixel, generate rays using Monte Carlo sampling:
    \begin{itemize}
        \item Sample pixel positions using \texttt{Sampler::getPixelSample}
        \item Generate differential rays via \texttt{Camera::generateDifferentialRay}
    \end{itemize}
    
    \item \textbf{Ray Tracing Loop}: Recursively trace rays up to \texttt{max\_depth} bounces:
    \begin{itemize}
        \item Test ray-scene intersection using \texttt{scene->intersect}
        \item Identify surface type via RTTI (IdealDiffusion or PerfectRefraction)
        \item Handle infinite lights for missed rays
    \end{itemize}
    
    \item \textbf{Surface Handling}:
    \begin{itemize}
        \item \textit{Diffuse surfaces}: Terminate and compute direct lighting
        \item \textit{Refractive surfaces}: Sample new direction and spawn continuation ray
        \item \textit{Other surfaces}: Terminate ray path
    \end{itemize}
    
    \item \textbf{Direct Lighting}: Compute illumination from point lights and environment maps with shadow testing and Monte Carlo integration
\end{enumerate}

\subsubsection{PerfectRefraction Material}

Implemented in \texttt{src/bsdf.cpp} (lines 77-123), this material simulates perfect specular refraction and total internal reflection:

\begin{enumerate}
    \item \textbf{Interface Detection}: Determine ray direction relative to surface normal:
    \begin{equation}
        \text{entering} = \cos\theta_i = \mathbf{n} \cdot \mathbf{w}_o > 0
    \end{equation}
    
    \item \textbf{IOR Adjustment}: Correct refractive index based on ray direction:
    \begin{equation}
        \eta_{\text{corrected}} = \begin{cases}
            \eta & \text{if entering} \\
            1/\eta & \text{if exiting}
        \end{cases}
    \end{equation}
    
    \item \textbf{Refraction Calculation}: Attempt to compute refracted direction using Snell's law:
    \begin{equation}
        \mathbf{w}_i = \text{Refract}(\mathbf{w}_o, \mathbf{n}_{\text{corrected}}, \eta_{\text{corrected}})
    \end{equation}
    
    \item \textbf{Total Internal Reflection}: If refraction fails (critical angle exceeded), perform perfect reflection:
    \begin{equation}
        \mathbf{w}_i = \text{Reflect}(\mathbf{w}_o, \mathbf{n}_{\text{corrected}})
    \end{equation}
\end{enumerate}

The material returns a delta distribution (pdf = 1.0) since it represents perfect specular interactions.

\subsection{Direct Lighting with Diffuse BRDF and Shadow Testing}

The direct lighting function is implemented in \texttt{IntersectionTestIntegrator::directLighting} (lines 146-213 in \texttt{src/integrator.cpp}). It computes illumination from both point lights and environment maps with visibility testing.

\subsubsection{Point Light Illumination}

For each point light in the scene:
\begin{enumerate}
    \item \textbf{Light Direction and Distance}: Compute normalized direction and distance to light:
    \begin{equation}
        \mathbf{l} = \text{Normalize}(\mathbf{p}_{\text{light}} - \mathbf{p}), \quad d = \|\mathbf{p}_{\text{light}} - \mathbf{p}\|
    \end{equation}
    
    \item \textbf{Shadow Testing}: Cast shadow ray from surface point to light source with $t_{\max} = d - \epsilon$. If intersected, add ambient term and skip to next light
    
    \item \textbf{Direct Lighting Computation}: For visible lights, compute radiance using diffuse BRDF:
    \begin{equation}
        L_d = \frac{\rho}{\pi} \cdot \Phi_{\text{light}} \cdot \max(0, \mathbf{n} \cdot \mathbf{l}) \cdot \frac{1}{d^2}
    \end{equation}
    where $\rho$ is the albedo from BSDF evaluation, $\Phi_{\text{light}}$ is the light flux
\end{enumerate}

\subsubsection{Environment Map Illumination}

For scenes with infinite area lights:
\begin{enumerate}
    \item \textbf{Monte Carlo Sampling}: Generate 30 samples from environment map using \texttt{InfiniteLight::sample}
    
    \item \textbf{Visibility Testing}: For each sample direction $\mathbf{w}_i$, cast shadow ray to test occlusion
    
    \item \textbf{Radiance Integration}: Accumulate unoccluded contributions:
    \begin{equation}
        L_{\text{env}} = \frac{1}{N}\sum_{i=1}^{N} \frac{\rho}{\pi} \cdot L_e(\mathbf{w}_i) \cdot \max(0, \mathbf{n} \cdot \mathbf{w}_i)
    \end{equation}
    where $N=30$ is the number of environment samples
\end{enumerate}

\subsection{Anti-Aliasing via Multi-Ray Sampling}

Anti-aliasing is implemented in the \texttt{IntersectionTestIntegrator::render} method (lines 41-77 in \texttt{src/integrator.cpp}) using stochastic sub-pixel sampling.

\begin{enumerate}
    \item \textbf{Samples Per Pixel (SPP)}: For each pixel $(dx, dy)$, generate \texttt{spp} rays with randomized sub-pixel positions
    
    \item \textbf{Sub-Pixel Jittering}: Use \texttt{Sampler::getPixelSample} to generate random positions within the pixel aperture $[dx, dx+1] \times [dy, dy+1]$
    
    \item \textbf{Ray Generation}: Create differential rays for each sample position via \texttt{Camera::generateDifferentialRay}
    
    \item \textbf{Monte Carlo Integration}: Accumulate radiance contributions:
    \begin{equation}
        L_{\text{pixel}} = \frac{1}{\text{spp}} \sum_{i=1}^{\text{spp}} L_i(\mathbf{r}_i)
    \end{equation}
    where $\mathbf{r}_i$ is the $i$-th ray sampled at a random sub-pixel position
    
    \item \textbf{Film Accumulation}: Each sample is committed to the film using \texttt{Film::commitSample}, which internally averages all contributions
\end{enumerate}

This approach effectively reduces aliasing artifacts by averaging multiple samples with different ray directions per pixel, smoothing out high-frequency discontinuities at geometric edges.
\section{Results}
\begin{figure}[h]
  \centering
  \includegraphics[width=\linewidth]{1.png}
  \caption{cbox with no refract}
\end{figure}

\begin{figure}[h]
  \centering
  \includegraphics[width=\linewidth]{2.png}
  \caption{cbox with refract}
\end{figure}

\begin{figure}[h]
  \centering
  \includegraphics[width=\linewidth]{3.png}
  \caption{environment light}
\end{figure}

\begin{figure}[h]
  \centering
  \includegraphics[width=\linewidth]{4.png}
  \caption{texture mapping}
\end{figure}
\end{document}

